\documentclass{article}
\usepackage{fancyref}
\usepackage{todonotes}

\title{Project Proposal \\ Using Agents and Genetic Algorithms to model, analyse and improve strategies for trading on an online auction site with feedback and reputation systems}
\author{Christopher Taylor \\ Department of Computer Science \\ University of Bath \\ ct345@bath.ac.uk}

\begin{document}
\maketitle
\todo{better title}
\listoftodos
\section{Problem Description}
\label{sec:problem-description}
Online auction websites - such as eBay - give users the opportunity to trade with eachother. Whilst this functionality is wonderful in terms of blah blah blah \todo{Talk about fraud on online marketplaces}

This project will create an abstract model of an online auction environment and use this to explore different strategies for both genuine and malicious users. This project spans mutliple Computer Science and Mathematical topics, including: Game Theory, Intelligent Agents and Genetic Algortihms.
\subsection{The Auction Model}
	\begin{figure}
		\label{fig:model-payoff}
		\caption{The Payoff Matrix for a transaction}
		\begin{tabular}{| l || c | c | c |}
			\hline
			A1,A2 & A2 Coop & A2 Defe & A2 Decl \\ \hline
			A1 Coop & 2,2 & -2,4 & 0,0 \\ \hline  
			A1 Defe & 4,-2 & -1, -1 & 0,0 \\ \hline
			A1 Decl & 0,0 & 0,0 & 0,0 \\ \hline
		\end{tabular}
	\end{figure}
\begin{itemize}
	\item N agents (representing users) are somehow selected or generated.
	\item The number of transactions for the entire simulation is randomly determined, and hidden from the agents.
	\item That many transactions runs, each one going as follows:
	\begin{itemize}
		\item Two random agents from the list are selected
		\item Each choose whether to Cooperate, Defect or Decline. An agent that chooses to cooperate goes into the transaction with good faith, while an agent that chooses to defect attempts to scam or cheat the other. Each agent also has the choice to decline a transaction. The payoff for these actions is defined in \fref{fig:model-payoff}.
		\item Each user then has the opportunity to leave feedback about the user they traded with (if the transaction actually went ahead - that is, if neither declined it.)
	\end{itemize}
	\item After all the transactions, each agent's score is computed.
	\item Then, a new generation of agents can be bred from the results (see \fref{sec:genetic-algorithms}) and analysis of the results can be performed.
\end{itemize}

\subsubsection{Justification for model}
\label{sec:justification-model}
As previously discussed, this model is very much an abstraction rather than a faithful representation of how a site may work. Things not factored in include:
\begin{itemize}
	\item Modelling of the actual content of transactions: the physical packaging and sending of goods, what the goods actually are, methods of payment and so on.s
	\item Relatedly, this includes different methods of defection - sending wrong/broken goods or not sending anything. Additionally, things such as a transaction falling through without malicious intent are not modelled.
	\item The payoffs for each person are simplified and uniform.
	\item Real users do not choose who to trade with at random.
\end{itemize}

Despite these limitations, I believe the model still has value as an analytical tool.

\todo{Expand justification}

\subsubsection{Game Theory Analysis}
\label{sec:game-theory}
Each individual transaction decision is typical of a Prisoner's Dilemma, and as such we can learn from the well-developed thought that surrounds that problem.

The Prisoner's Dilemma is usually explained thusly: two people suspected of committing some crime have been arrested and are being interrogated by the police. The police officer in question doesn't have sufficient evidence to arrest them on the crime they have committed, but can arrest them on a lesser charge. The officer places them in seperate rooms and offers them a deal - they can either stay quiet (cooperate), or confess and incriminate their partner (defect).
\begin{itemize}
	\item If both of the suspects stay quiet, they will both be charged with the lesser offense and given a short prison sentence.
	\item If one of the suspects talks, that suspect walks free while the other suffers a long prison sentence.
	\item However, if both of them talk, neither can reap the benefits from being a witness and both will suffer a long sentence.
\end{itemize}
	\begin{figure}
		\label{fig:generic-prisoners-dilemma}
		\caption{The payoff matrix for a genetic prisoner's dilemma}
		\begin{tabular}{| l || c | c | c |}
			\hline
			A1,A2 & A2 Coop & A2 Defe \\ \hline
			A1 Coop & -1,-1 & -7,0 \\ \hline  
			A1 Defe & 0,-7 & -3,-3  \\ \hline
		\end{tabular}
	\end{figure}

A single instance of a Prisoner's Dilemma is a solved problem from a Game Theory perspective - it's Nash Equilibrium is for both players to defect, despite the fact that, if both had cooperated, they would have gotten a shorter sentence.

Repeated instances of the Prisoner's Dilemma is a more complicated issue - that is, when the same players undertake $n$ number of games, with full memory of all the previous games. While, for a fixed and known $n$, the game has the same nash equilibrium (that is, to always defect) - the repeated nature of the game gives an opportunity for counterplay and changing decisions.

The most simple strategies are a naive \emph{always cooperate} or \emph{always defect} strategy, but there are other - more complicated - strategies, which may or may not produce better results on average. Some of these include:
\begin{description}
	\item[Random] - Chooses an option at random. Sometimes $P(defect) = P(cooperate) = 0.5$, but the probabilites can theoretically be anything.
	\item[Tit for Tat] - Cooperate in the first round, then repeat the opponent's move every round after that. (I.E. defect against an opponent who defected last round.)
	\item[Tit for Two Tat] - Cooperate unless the opponent defects twice in a row, then defect
	\item[Grim Trigger] - Cooperate until opponent defects, then defect forever.
\end{description}

\todo{How this problem is similar/different from this standard prisoner's dilemma}

\subsubsection{Psychology and Sociology of cooperation}
\label{sec:psychology-sociology-cooperation}
\todo{Psychology and Sociology of cooperation, including lots of references to Liars and Outliers. \\ Read this as "why all the game theory is wrong when applied to humans."}

\subsection{Genetic Algorithms}
\label{sec:genetic-algorithms}
\todo{Genetic Algorithms: what they are, why I'm using them.}

\section{Requirements}
\label{sec:requirements}
\todo{Requirements.}

\section{Project Plan}
\label{sec:project-plan}
\todo{Project Plan}

\end{document}
