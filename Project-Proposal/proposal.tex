\documentclass{article}
\usepackage{fancyref}

\title{Project Proposal \\ Using Agents and Genetic Algorithms to model, analyse and improve strategies for trading on an online auction site with feedback and reputation systems}
\author{Christopher Taylor \\ Department of Computer Science \\ University of Bath \\ ct345@bath.ac.uk}

\begin{document}
\maketitle
\section{Problem Description}
\label{sec:problem-description}
Online auction websites - such as eBay - give users the opportunity to trade with eachother. Whilst this functionality is wonderful in terms of blah blah blah

This project will create an abstract model of an online auction environment and use this to explore different strategies for both genuine and malicious users. This project spans mutliple Computer Science and Mathematical topics, including: Game Theory, Intelligent Agents and Genetic Algortihms.
\subsection{The Auction Model}
	\begin{figure}
		\label{fig:model-payoff}
		\caption{The Payoff Matrix for a transaction}
		\begin{tabular}{| l || c | c | c |}
			\hline
			A1,A2 & A2 Coop & A2 Defe & A2 Decl \\ \hline
			A1 Coop & 2,2 & -2,4 & 0,0 \\ \hline  
			A1 Defe & 4,-2 & -1, -1 & 0,0 \\ \hline
			A1 Decl & 0,0 & 0,0 & 0,0 \\ \hline
		\end{tabular}
	\end{figure}
\begin{itemize}
	\item N agents (representing users) are somehow selected or generated.
	\item The number of transactions for the entire simulation is randomly determined, and hidden from the agents.
	\item That many transactions runs, each one going as follows:
	\begin{itemize}
		\item Two random agents from the list are selected
		\item Each choose whether to Cooperate, Defect or Decline. An agent that chooses to cooperate goes into the transaction with good faith, while an agent that chooses to defect attempts to scam or cheat the other. Each agent also has the choice to decline a transaction. The payoff for these actions is defined in \fref{fig:model-payoff}.
		\item Each user then has the opportunity to leave feedback about the user they traded with (if the transaction actually went ahead - that is, if neither declined it.)
	\end{itemize}
	\item After all the transactions, each agent's score is computed.
	\item Then, a new generation of agents can be bred from the results (see \fref{sec:genetic-algorithms}) and analysis of the results can be performed.
\end{itemize}

\subsubsection{Justification for model}
\label{sec:justification-model}
As previously discussed, this model is very much an abstraction rather than a faithful representation of how a site may work. Things not factored in include:
\begin{itemize}
	\item Modelling of the actual content of transactions: the physical packaging and sending of goods, what the goods actually are, methods of payment and so on.s
	\item Relatedly, this includes different methods of defection - sending wrong/broken goods or not sending anything. Additionally, things such as a transaction falling through without malicious intent are not modelled.
	\item The payoffs for each person are simplified and uniform.
	\item Real users do not choose who to trade with at random.
\end{itemize}

\subsubsection{Game Theory Analysis}
\label{sec:game-theory}

\subsection{Genetic Algorithms}
\label{sec:genetic-algorithms}

\section{Requirements}
\label{sec:requirements}

\section{Project Plan}
\label{sec:project-plan}

\end{document}
